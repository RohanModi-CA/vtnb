\documentclass[12pt]{article}
\input{/home/thinkpad/Documents/FileFolder/setup/preamble.tex}
\pagestyle{fancy}


%%%%%%%%%%%%%%%%%%%%%%%% START %%%%%%%%%%%%%%%%%%%%%%%%%%%%%
\begin{document} % \pagenumbering{arabic}
\fancyhead[C]{\scshape { {} - {Problem Set }} } 
\begin{center} \section*{$\boxed{\text{ Question  }}$} \end{center} 










%%% vtnb start %%%
\begin{lstlisting}[frame=shadowbox]
import numpy as np # import built-in librairies (super useful)

import matplotlib.pyplot as plt

# make graphics show up in Jupyter notebook

%matplotlib inline 

import os

import pylab



def data_generator(mean, std, n_measurements = 500, n_samples = 20): # function name and arguments

    """

    Generates an array of measurements from a standard distribution.



    Parameters

    ----------

    mean : float

           Desired mean value of the measurements.

           

    std : float

          Desired standard deviation of the measurements.

          

    n_measurements : int, optional

                     Number of separate measurements. Default is 500 measurements.



    n_samples : int, optional

                Number of samples taken per measurement. Default is 20 samples. 

                per measurement.





    Returns

    -------

    data : ndarray, shape (n_measuremets, n_samples)

           Array representing the experimental data. Each measurement 

           (composed of many samples) is a row of this array:

                     -----------------------------------

            meas1    | sample0 | sample1 | sample2 | ...

            meas2    | sample0 | sample1 | sample2 | ...

            meas3    | sample0 | sample1 | sample2 | ...

             ...



    """

    # The following array has n_measurements rows, and n_samples columns

    return np.random.normal(loc = mean, scale = std, size = (n_measurements, n_samples))
\end{lstlisting}
%%% vtnb end %%%


%%% vtnb start %%%
\begin{lstlisting}[frame=shadowbox]
data = np.genfromtxt("/home/rohan/Téléchargements/buggsa.csv", delimiter=",", invalid_raise=False)

print(data)
\end{lstlisting}
%%% vtnb end %%%


Yes, it does make sense to transpose this.


%%% vtnb start %%%
\begin{lstlisting}[frame=shadowbox]
data = np.transpose(data)
\end{lstlisting}
%%% vtnb end %%%


So now we handle the nan and put each row into its own python list, run it through numpy again, and put the means into another list.


%%% vtnb start %%%
\begin{lstlisting}[frame=shadowbox]
data_l = []

row = 0

for i in data:

    data_c = []

    for j in i:

        if not np.isnan(j):

            data_c.append(j)

    data_l.append(data_c)

    row += 1



#print(np.array(data_l))



means = []

stds = []

for i in data_l:

    means.append(float(np.mean(np.array(i))))

    stds.append(float(np.std(np.array(i),ddof=1)))



print(means, "\n", stds)
\end{lstlisting}
%%% vtnb end %%%


These values do agree with manual calculations! We will choose the 2nd distance, 40cm to generate our data.


%%% vtnb start %%%
\begin{lstlisting}[frame=shadowbox]
sim_data = data_generator(means[1], stds[1])
\end{lstlisting}
%%% vtnb end %%%


We will now plot this as a histogram:


%%% vtnb start %%%
\begin{lstlisting}[frame=shadowbox]
flattened = (sim_data.flatten())

plt.hist(flattened, bins=10);
\end{lstlisting}
%%% vtnb end %%%


That doesn't look very Gaussian! Let's change the amount of bins, and see something much more pleasant:


%%% vtnb start %%%
\begin{lstlisting}[frame=shadowbox]
counts, bins, _ = plt.hist(flattened, bins=100); # save the outputs, into variables, of pyplot.hist, ignoring the "patches" return object.
\end{lstlisting}
%%% vtnb end %%%


There we go! Let's investigate the mean and standard deviation of that.


%%% vtnb start %%%
\begin{lstlisting}[frame=shadowbox]
print(f"The mean is {np.mean(flattened)}, and the standard deviation is: {np.std(flattened, ddof=1)}.")
\end{lstlisting}
%%% vtnb end %%%


%%% vtnb start %%%
\begin{lstlisting}[frame=shadowbox]
# Save the counts (heights of the bars) to a file

np.savetxt("histogram_counts.txt", counts)



# Save the bin edges (edges of the bins) to another file if needed

np.savetxt("histogram_bins.txt", bins)
\end{lstlisting}
%%% vtnb end %%%


We will now be plotting a histogram of all of the means from each trial. So we will first make a list of means.


%%% vtnb start %%%
\begin{lstlisting}[frame=shadowbox]
sd_means = []

for i in sim_data:

    sd_means.append(np.mean(i))





sdm_count, sdm_bins, _ = plt.hist(sd_means, bins=25);

np.savetxt("sdm_count.txt", bins)

np.savetxt("sdm_bins.txt", bins)
\end{lstlisting}
%%% vtnb end %%%


The means and standard deviations are as follows:


%%% vtnb start %%%
\begin{lstlisting}[frame=shadowbox]
mean_means = np.mean(sd_means)

mean_stds = np.std(sd_means, ddof=1)

print(f"The mean of the means is: {mean_means}, and the standard deviation is: {mean_stds}")
\end{lstlisting}
%%% vtnb end %%%


Okay, quick detour, let's find the area under our histogram so that we can normalize our Gaussian.


%%% vtnb start %%%
\begin{lstlisting}[frame=shadowbox]
def area_of_bins(counts, bins):

    area = 0

    width = bins[1] - bins[0]

    for i in counts:

        area += width * i

    # the width is constant if you specify bins= as an integer, within plt.hist(), as said in documentation

    return float(area)

    

#area_of_bins(sdm_count, sdm_bins);
\end{lstlisting}
%%% vtnb end %%%


Great! With that out of the way, let me start getting my values for my Gaussian. Let's make a function:


%%% vtnb start %%%
\begin{lstlisting}[frame=shadowbox]
def gaussian(x, mu, sigma):

    return (1)/(sigma * np.sqrt(2*np.pi)) * np.exp(    -1 * ( ((x-mu)**2)/( 2 * (sigma**2) ) )  )
\end{lstlisting}
%%% vtnb end %%%


%%% vtnb start %%%
\begin{lstlisting}[frame=shadowbox]
outputs = []

for i in np.linspace(2, 3, num=500):

    outputs.append(gaussian(i, mean_means, mean_stds))
\end{lstlisting}
%%% vtnb end %%%


%%% vtnb start %%%
\begin{lstlisting}[frame=shadowbox]

\end{lstlisting}
%%% vtnb end %%%


%%%%%%%%%%%%%%%%%%%%%%%%  END  %%%%%%%%%%%%%%%%%%%%%%%%%%%%%
\end{document}
